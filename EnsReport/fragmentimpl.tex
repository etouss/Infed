\section{Implementation}
\label{sec:implementation}
To obtain efficient solving times, we implemented several optimisation of the procedures described in the previous section.
We describe these optimisations in the next subsections.
It is important to understand that our decision procedure being based upon an SMT solver, we want our output formula to be as friendly as possible.
However, we try to keep our computation time as small as possible too.
The challenge is to understand when we should increase our computation time in order to decrease the computation time of SMT solvers.
As shown by our benchmark, the SMT solvers are very efficient when dealing with standards operations (for instance, convert a formula in CNF), hence we do not modify our formula form.
However SMT solvers computation time increases with the number of atoms in the input formula. As a consequence it is interesting to reduce this number by
our reduction procedure, if it does not increase too much our computation time (see Section \ref{ssec:bench}).

\subsection{Unitary Bag Optimisation}
Many atoms are added in steps 2-4 due to $L_{bag}$ and $L_{mix}$; more precisely, their number is linear in the number of variable in $V_{int}$ and $V_{bag}$.
To reduce this number, we rewrite the formulas in the left side below to the right side formulas which add less variables and conjunctions in steps 2-4:
\begin{center}
\begin{tabular}{lcl}
  $\bagone{a} \subseteq x$ & $\sim_{\mathit{sat}}$ & $a \in x$ \\
  $\bagone{a} \nsubseteq x$ & $\sim_{\mathit{sat}}$ & $a \notin x$ \\
  $max(\bagone{a})$ & $=$ & $a$ \\
  $min(\bagone{a})$ & $=$ & $a$ \\
\end{tabular}
\end{center}
In $S4$,  $a \in x$ does not add any conjunction to the formula, hence the output have fewer atoms than the equi-satisfiable one with $\bagone{a} \subseteq x$.
As $a \notin x$ does not add any variable to $V_{int}$ fewer atoms will be conjuncted during $S4$ than with $\bagone{a} \nsubseteq x$
Same can be said about $max(\bagone{a})$ and $min(\bagone{a})$ where step 2 adds a fresh variable to $V_{int}$ and step 4 conjuncts many atoms.
%In conclusion, it might be interesting to rewrite those atoms.
Notice that the rewriting can be done without any increase of computation time, as it is just a different rewriting of an already existing literal.

\subsection{Bag variables creation}
In step 2.4, every bag atom $L_{bag}$ of $\tilde{F}_{2.3}$ using more than two bag variables for operations $\neq, \subseteq, \nsubseteq$ and more than three variables for operation $=$ are iteratively rewritten to be reduced to the bag atoms in $\QFBILIA_{pure}$.
However, this rewriting adds many bag variable which cost a lot of computation time, as we will have to deal with more bags.
Let define a function $apply : (T_{bag},V_{int}) \rightarrow T_{int}$ by:
\begin{flalign}
  apply(x,a)  &=    w_{a,x}    \\
  apply(T_{bag}^1 \cup T_{bag}^2,a)  &=  max(apply(T_{bag}^1,a),apply(T_{bag}^2,a))    \\
  apply(T_{bag}^1 \cap T_{bag}^2,a)  &=  min(apply(T_{bag}^1,a),apply(T_{bag}^2,a))    \\
  apply(T_{bag}^1 \uplus T_{bag}^2,a)  &=  apply(T_{bag}^1,a)+apply(T_{bag}^2,a)    \\
  apply(T_{bag}^1 \setminus T_{bag}^2,a)  &=  max(0,apply(T_{bag}^1,a)-apply(T_{bag}^2,a))    \\
\end{flalign}
Then we can extend S4:
$$S4(T_{bag}^1 = T_{bag}^2) \triangleq \underset{t \in V^3_{ele}}{\bigwedge}(apply(T_{bag}^1,t) = apply(T_{bag}^2,t))$$
while preserving equi-satisfiability.
Same can be done for $\neq, \subseteq, \nsubseteq$ operators.
This rewriting allows us to skip the step 2.4.

\subsection{Integer Element Optimisation}
A \QFBILIA\ formula might have pure integer variables, i.e., integer variables that are not involve in any bag, mix or extremum literals.
Such variable does not have to be consider during the counting abstraction as they do not constrain any bag, hence we can reduce the cardinal of $V^3_{ele}$

\begin{mydef}
  Let a formula $F$ in \QFBILIA\ let two variable $u$, $v$ in $V_{int}(F)\cup V_{bag}(F)$ then $u$ operate with $v$, denote
  $u \backsimeq v$, iff exists $l$ in $L(F)$ where $u$ and $v$ are in $V_{int}(l) \cup V_{bag}(l)$.
  $$u \backsimeq v \Longleftrightarrow \exists l \in L(F), u,v \in V_{int}(l)\cup V_{bag}(l)$$
\end{mydef}
\begin{mydef}
  Let a formula $F$ in \QFBILIA\ , $V_{ope}(F)$ is the set of integer variable:
  $$V_{ope}(F) = \left\{a \in V_{int}(F) | \exists x \in V_{bag}(F), a \backsimeq y \right\}\cup V_{ext}$$
  and
  $$<V_{ope}(F)> = \left\{a \in V_{ope}(F) | \exists b \in V_{ope}(F), a \backsimeq b \right\}$$
\end{mydef}

\begin{myprop}
If $V^3_{ele}$ is redefined (step 3-4) as :$V^3_{ele} = <V_{ope}(F)>$ then
(i) Property 3 is still valid, and
(ii) Property 4 is  still valid.
\end{myprop}

\begin{proof}
The proof is the same as before because the only difference is the absence of useless $w$.
\end{proof}


% Our decision procedure being based on Bags of integer and linear integer arithmetic, the integer variable $V_{int}(F)$ of the formula
% can be split in 2 different type. Let us denote $V_{ele}(F)$ the set of integer variable that interact with bags in $F$.
% It's important to understand that $V_{ele}(F)$ contains any integer variable that interact directly with bags thanks to $L_{mix}$ atoms
% but also any integer variable which interact with an element form $V_{ele}(F)$ thanks to $L_{int}$ atoms.
% ie, $V_{ele}(F)$ is closed for $L_{int}$ atoms.
% Finally $V_{ele}(F)$ can be use during step 3 and 4, indeed we can apply the count abstraction only amoung those integer
% without loosing the equi-satisfiability property.
% \\Such $V_{ele}(F)$ have to be construct thanks to an union-find algorithm. Hence if the input formula doesn't contain any integer which isn't in $V_{ele}(F)$
% our compution time increases without any benefit for the output formula. For this we implemented an option in the programme to enable or disable it.

\subsection{Benchmark}
\label{ssec:bench}

Our solvers takes as input problems written in the SMTLIB format~\cite{SMTLIB10} for the theory of bags and sets.

We applied it on a benchmark of 295 problems generated from verification conditions generated by~\cite{enea2014compositional}
for programs manipulating data structures
like lists, binary search trees, AVL, skip lists,\ldots.
All problems take less than 1 sec to be reduced to the base theory (QFLIA or QFUFLIA).
%
%Z3
%CVC4
%API Z3
%Int element opti
%UFLIA
%LIA

% During the third and fourth steps of the descion procedure we didn't take into account that the formula
% $\tilde{F}^2$ might have integer variable that don't participate in any


% In order to understand this section we have to considerate the cost of each step and identify from where that cost come from.
% %\label{sec:fimpl}
% \subsection{Translation to \QF_BILIA\ Complexity}
% The complexity of the nnf function is linear among atoms of the formula indeed we only have to treat each of them once.
% However as we see, rewritting of ite terms duplicate the condition.
% As a conclusion we should avoid ite as much as possible as their rewritting cost a lot.
%
% \subsection{Translation to $\QFBILIA_{pure}$ Complexity}
% Once again each atoms of the formula will be treat once then it's linear.
%
% \subsection{Counting Abstraction Complexity}
% This step is expensive, indeed we have to add a variable for each bag and for each
%
%
% It's easy to notice that their is a difference in our algorithm between the rewritting of the following formula
% which are equi-satisfiable.
% \begin{align}
% \bagone{a} \subseteq x & \sim_{\mathit{sat}} & a \in x \\
% \end{align}
% In order to optimise we can rewrite many $\llbracket a \rrbracket$ constraint to avoid costly operation:
% \begin{center}
% \begin{tabular}{l c l}
% $\llbracket a \rrbracket \subseteq x$ & $\Rightarrow$ & $a \in x$\\
% $\llbracket a \rrbracket \nsubseteq x$ & $\Rightarrow$ & $a \notin x$\\
% $min(\llbracket a \rrbracket)$ & $\Rightarrow$ & $a$\\
% $max(\llbracket a \rrbracket)$ & $\Rightarrow$ & $a$\\
% \end{tabular}
% \end{center}
