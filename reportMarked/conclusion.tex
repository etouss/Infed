%!TEX root = main.tex

\section{Conclusion}
%\label{sec:concl}

%\\ The main constraint of our logic it's the fact that we only work on $bag\_int$ in deed when we
%try to produce same kind of result on element $e \in E$ where $E$ is set, which is solvable under a theorie $E$.
%We are dealing with an issue: First of all $T$ have to implement $=$ , and order to work with everything.
%However even if $T$ does the main optimisation to loose Non-deterministic rewritting can't be done
%because element of $E$ canno't be mix with $int$ so we won't be able to solve
%\\
%$\llbracket a \rrbracket$ ; $baga$ ; $a\in baga$ and $\forall i \in \mathbb{Z} ( i \neq a  \Rightarrow i \notin baga )$
%\\
%But as notice in section X: $\llbracket a \rrbracket$ may be avoid in most formula thanks to $\in$ and other trick
%\\ UF\_FREE formula won't be avoid cause their rewriting involve $(a = b) \Rightarrow \bigwedge_{x \in \nu_{bag}} \; xnba = xnbb $
%Where $a$ , $b$ are $E$ element and $xnba$ , $xnbb$  are $int$
%But $X : E->int$ can be define.
We defined the logic \QFBILIA\ to express constraints over bags of integers. 
We then presented two decision procedures for \QFBILIA. 
The main flow of our work is that we were only able to be determinist because we were working with bags of integers.
Moreover the efficiency of our decision procedures rely on the SMT solvers upon which we are working.
